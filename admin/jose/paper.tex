\documentclass[10pt,a4paper,onecolumn]{article}
\usepackage{marginnote}
\usepackage{graphicx}
\usepackage{xcolor}
\usepackage{authblk,etoolbox}
\usepackage{titlesec}
\usepackage{calc}
\usepackage{tikz}
\usepackage{hyperref}
\hypersetup{colorlinks,breaklinks,
            urlcolor=[rgb]{0.0, 0.5, 1.0},
            linkcolor=[rgb]{0.0, 0.5, 1.0}}
\usepackage{caption}
\usepackage{tcolorbox}
\usepackage{amssymb,amsmath}
\usepackage{ifxetex,ifluatex}
\usepackage{seqsplit}
\usepackage{fixltx2e} % provides \textsubscript
\usepackage[
  backend=biber,
%  style=alphabetic,
%  citestyle=numeric
]{biblatex}
\bibliography{paper.bib}



% --- Page layout -------------------------------------------------------------
\usepackage[top=3.5cm, bottom=3cm, right=1.5cm, left=1.0cm,
            headheight=2.2cm, reversemp, includemp, marginparwidth=4.5cm]{geometry}

% --- Default font ------------------------------------------------------------
% \renewcommand\familydefault{\sfdefault}

% --- Style -------------------------------------------------------------------
\renewcommand{\bibfont}{\small \sffamily}
\renewcommand{\captionfont}{\small\sffamily}
\renewcommand{\captionlabelfont}{\bfseries}

% --- Section/SubSection/SubSubSection ----------------------------------------
\titleformat{\section}
  {\normalfont\sffamily\Large\bfseries}
  {}{0pt}{}
\titleformat{\subsection}
  {\normalfont\sffamily\large\bfseries}
  {}{0pt}{}
\titleformat{\subsubsection}
  {\normalfont\sffamily\bfseries}
  {}{0pt}{}
\titleformat*{\paragraph}
  {\sffamily\normalsize}


% --- Header / Footer ---------------------------------------------------------
\usepackage{fancyhdr}
\pagestyle{fancy}
\fancyhf{}
%\renewcommand{\headrulewidth}{0.50pt}
\renewcommand{\headrulewidth}{0pt}
\fancyhead[L]{\hspace{-0.75cm}\includegraphics[width=5.5cm]{C:/Users/smaso/AppData/Local/R/win-library/4.4/rticles/rmarkdown/templates/joss/resources/JOSS-logo.png}}
\fancyhead[C]{}
\fancyhead[R]{}
\renewcommand{\footrulewidth}{0.25pt}

\fancyfoot[L]{\footnotesize{\sffamily Garrison, (). Data Science for
Psychologists (DS4P): A Modular, Open-Source Learning Module for
Transparent and Reproducible Data Analysis in
Psychology. \textit{Journal of Open Source Software}, (), . \href{https://doi.org/}{https://doi.org/}}}


\fancyfoot[R]{\sffamily \thepage}
\makeatletter
\let\ps@plain\ps@fancy
\fancyheadoffset[L]{4.5cm}
\fancyfootoffset[L]{4.5cm}

% --- Macros ---------

\definecolor{linky}{rgb}{0.0, 0.5, 1.0}

\newtcolorbox{repobox}
   {colback=red, colframe=red!75!black,
     boxrule=0.5pt, arc=2pt, left=6pt, right=6pt, top=3pt, bottom=3pt}

\newcommand{\ExternalLink}{%
   \tikz[x=1.2ex, y=1.2ex, baseline=-0.05ex]{%
       \begin{scope}[x=1ex, y=1ex]
           \clip (-0.1,-0.1)
               --++ (-0, 1.2)
               --++ (0.6, 0)
               --++ (0, -0.6)
               --++ (0.6, 0)
               --++ (0, -1);
           \path[draw,
               line width = 0.5,
               rounded corners=0.5]
               (0,0) rectangle (1,1);
       \end{scope}
       \path[draw, line width = 0.5] (0.5, 0.5)
           -- (1, 1);
       \path[draw, line width = 0.5] (0.6, 1)
           -- (1, 1) -- (1, 0.6);
       }
   }

% --- Title / Authors ---------------------------------------------------------
% patch \maketitle so that it doesn't center
\patchcmd{\@maketitle}{center}{flushleft}{}{}
\patchcmd{\@maketitle}{center}{flushleft}{}{}
% patch \maketitle so that the font size for the title is normal
\patchcmd{\@maketitle}{\LARGE}{\LARGE\sffamily}{}{}
% patch the patch by authblk so that the author block is flush left
\def\maketitle{{%
  \renewenvironment{tabular}[2][]
    {\begin{flushleft}}
    {\end{flushleft}}
  \AB@maketitle}}
\makeatletter
\renewcommand\AB@affilsepx{ \protect\Affilfont}
%\renewcommand\AB@affilnote[1]{{\bfseries #1}\hspace{2pt}}
\renewcommand\AB@affilnote[1]{{\bfseries #1}\hspace{3pt}}
\makeatother
\renewcommand\Authfont{\sffamily\bfseries}
\renewcommand\Affilfont{\sffamily\small\mdseries}
\setlength{\affilsep}{1em}


\ifnum 0\ifxetex 1\fi\ifluatex 1\fi=0 % if pdftex
  \usepackage[T1]{fontenc}
  \usepackage[utf8]{inputenc}

\else % if luatex or xelatex
  \ifxetex
    \usepackage{mathspec}
  \else
    \usepackage{fontspec}
  \fi
  \defaultfontfeatures{Ligatures=TeX,Scale=MatchLowercase}

\fi
% use upquote if available, for straight quotes in verbatim environments
\IfFileExists{upquote.sty}{\usepackage{upquote}}{}
% use microtype if available
\IfFileExists{microtype.sty}{%
\usepackage{microtype}
\UseMicrotypeSet[protrusion]{basicmath} % disable protrusion for tt fonts
}{}

\usepackage{hyperref}
\hypersetup{unicode=true,
            pdftitle={Data Science for Psychologists (DS4P): A Modular, Open-Source Learning Module for Transparent and Reproducible Data Analysis in Psychology},
            pdfborder={0 0 0},
            breaklinks=true}
\urlstyle{same}  % don't use monospace font for urls
\usepackage{graphicx,grffile}
\makeatletter
\def\maxwidth{\ifdim\Gin@nat@width>\linewidth\linewidth\else\Gin@nat@width\fi}
\def\maxheight{\ifdim\Gin@nat@height>\textheight\textheight\else\Gin@nat@height\fi}
\makeatother
% Scale images if necessary, so that they will not overflow the page
% margins by default, and it is still possible to overwrite the defaults
% using explicit options in \includegraphics[width, height, ...]{}
\setkeys{Gin}{width=\maxwidth,height=\maxheight,keepaspectratio}
\IfFileExists{parskip.sty}{%
\usepackage{parskip}
}{% else
\setlength{\parindent}{0pt}
\setlength{\parskip}{6pt plus 2pt minus 1pt}
}
\setlength{\emergencystretch}{3em}  % prevent overfull lines
\setcounter{secnumdepth}{0}
% Redefines (sub)paragraphs to behave more like sections
\ifx\paragraph\undefined\else
\let\oldparagraph\paragraph
\renewcommand{\paragraph}[1]{\oldparagraph{#1}\mbox{}}
\fi
\ifx\subparagraph\undefined\else
\let\oldsubparagraph\subparagraph
\renewcommand{\subparagraph}[1]{\oldsubparagraph{#1}\mbox{}}
\fi


% tightlist command for lists without linebreak
\providecommand{\tightlist}{%
  \setlength{\itemsep}{0pt}\setlength{\parskip}{0pt}}





\title{Data Science for Psychologists (DS4P): A Modular, Open-Source
Learning Module for Transparent and Reproducible Data Analysis in
Psychology}

        \author[1]{S. Mason Garrison\footnote{corresponding author}}
    
      \affil[1]{Department of Psychology, Wake Forest University, North
Carolina, USA}
  \date{\vspace{-5ex}}

\begin{document}
\maketitle

\marginpar{
  %\hrule
  \sffamily\small

  {\bfseries DOI:} \href{https://doi.org/}{\color{linky}{}}

  \vspace{2mm}

  {\bfseries Software}
  \begin{itemize}
    \setlength\itemsep{0em}
    \item \href{}{\color{linky}{Review}} \ExternalLink
    \item \href{}{\color{linky}{Repository}} \ExternalLink
    \item \href{}{\color{linky}{Archive}} \ExternalLink
  \end{itemize}

  \vspace{2mm}

  {\bfseries Submitted:} \\
  {\bfseries Published:} 

  \vspace{2mm}
  {\bfseries License}\\
  Authors of papers retain copyright and release the work under a Creative Commons Attribution 4.0 International License (\href{http://creativecommons.org/licenses/by/4.0/}{\color{linky}{CC-BY}}).
}

\section{Summary}\label{summary}

\textbf{Data Science for Psychologists (DS4P)} is an open, GitHub-hosted
computational learning module developed to support data science
instruction within psychology. Initially launched as a graduate course
during the COVID-19 pandemic, DS4P will expand to include undergraduate
instruction beginning in Spring 2026 at Wake Forest University.

This multimedia module serves as a flipped textbook, combining embedded
video lectures, annotated code examples, readings, slide decks, and
hands-on labs. Its instructional approach emphasizes reproducibility and
transparency in psychological research. No prior programming or advanced
mathematical background is assumed.

DS4P makes extensive use of the R ecosystem---including R, R Markdown,
Git, and GitHub---and incorporates open educational resources from the
broader data science community, including \emph{STAT 545}
(\textbf{bryan\_stat545?}), \emph{Happy Git with R}
(\textbf{bryan\_git?}), and \emph{Data Science in a Box}
(\textbf{cetinkaya2020datascience?}). It is designed to build student
capacity for real-world, open science workflows in the behavioral
sciences.

\section{Statement of need}\label{statement-of-need}

Despite increasing attention to data literacy in psychology, most
students receive limited exposure to tools and practices that support
reproducible workflows. DS4P addresses this need by offering an
accessible, semester-length learning module that introduces data science
through domain-relevant applications in psychology. Unlike generic
``learn-to-code'' resources, DS4P's materials are embedded in the
context of psychological inquiry, allowing students to ``code to
learn.''

The course's availability as an open, Creative Commons--licensed
resource enables reuse, remixing, and adoption by instructors across
institutions. By integrating essential computing tools with
discipline-specific training, DS4P contributes to closing the gap
between data science and psychological science education.

\section{Module Description}\label{module-description}

DS4P is organized into modular units that together span an entire
semester (15+ weeks). Each week's unit includes:

\begin{itemize}
\tightlist
\item
  Readable narrative content with embedded R code and live output
\item
  R Markdown files modeling reproducible research documents
\item
  Interactive activities and labs with real datasets
\item
  Pre-recorded lectures and annotated slide decks
\item
  GitHub repositories for collaborative exercises and version control
\end{itemize}

The content scaffolds from introductory topics (e.g., data wrangling,
visualization) to more advanced material (e.g., modeling,
reproducibility workflows), culminating in independent data analysis
projects. All materials are openly licensed and regularly maintained.

\section{Instructional Design}\label{instructional-design}

The instructional design reflects research-based pedagogical practices:

\begin{itemize}
\tightlist
\item
  \textbf{Worked-example effect}: Code and output are presented
  alongside explanatory narrative, guiding students through reasoning
  processes step by step.
\item
  \textbf{Literate programming}: R Markdown is used throughout, teaching
  students to produce integrated code-and-narrative documents.
\item
  \textbf{Cognitive load management}: Multimedia content supports
  dual-channel learning, and lessons are segmented by concept.
\item
  \textbf{Authentic practice}: Students interact with real psychological
  datasets and research questions using the same tools as professionals.
\end{itemize}

The course also trains students in \emph{version control},
\emph{collaborative development}, and \emph{open science norms} using
Git and GitHub---tools not commonly integrated into undergraduate
psychology instruction.

\section{Experience of Use}\label{experience-of-use}

DS4P has been taught to multiple cohorts of graduate students since
2020, with positive reception and successful student projects applying
reproducible workflows. Beginning in Spring 2026, the course will also
be implemented for undergraduates. The modular structure facilitates
instructor customization and reuse.

\section{Availability}\label{availability}

The \texttt{DS4P} module is hosted at:
\url{https://github.com/mason-garrison/DS4P}\\
It is available under a Creative Commons Attribution-ShareAlike 4.0
International License for all narrative, instructional, and non-video
content. Video content is licensed under a Standard YouTube License, and
is available on the \href{https://www.youtube.com/@smasongarrison}{S.
Mason Garrison YouTube channel}.

\section{References}\label{references}

\end{document}
